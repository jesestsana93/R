\documentclass{article}
\usepackage[utf8]{inputenc} %para acentos y ñ
\usepackage[letterpaper,lmargin=2cm,rmargin=2cm, top=3cm, bottom=3cm ]{geometry} %margen de la hoja
\usepackage[spanish]{babel} % Español

\usepackage{amsmath}		% Soporte de símbolos adicionales (matemáticas)
\usepackage{amssymb}		% También se incluyen algunos entornos adicionales para matemáticas
\usepackage{amsfonts}
\usepackage{Sweave}
\begin{document}
\Sconcordance{concordance:soluTarea1.tex:soluTarea1.Rnw:%
1 8 1 1 0 7 1 1 2 1 0 2 1 6 0 1 2 15 1 1 2 1 0 1 1 8 0 2 1 9 0 1 2 9 1 %
1 2 1 0 7 1 20 0 1 2 4 1 1 2 7 0 1 1 7 0 1 2 2 1 1 2 8 0 1 2 1 1 1 2 8 %
0 1 2 2 1 1 2 1 0 1 1 6 0 1 2 1 1 1 2 8 0 1 2 1 1}


1. Calcule la siguiente suma
$$\sum_{i=1}^{25} \frac{2^i}{i} + \frac{3^i}{i^2}$$

En R se hace de la siguiente manera.

\begin{Schunk}
\begin{Sinput}
> i<-1:25
> suma<-sum((2^i)/i+(3^i)/(i^2))
> print(suma)
\end{Sinput}
\begin{Soutput}
[1] 2129170437
\end{Soutput}
\end{Schunk}

2.Suponga que 

\[
B = \left( 
\begin{array}{ccc}
	1 & 1 & 3 \\
	5 & 2 & 6 \\
	-2 & -1 & -3
\end{array}
\right)
\]

a) Comprobar que $A^3=0$, donde 0 es una matriz de 3x3 con todas las entradas igual a 0.\\
b) Reemplaza la tercera columna de A por la suma de la segunda y tercera columna.

\begin{Schunk}
\begin{Sinput}
> A<-matrix(c(1,5,-2,1,2,-1,3,6,-3),3,3)
> A%*%A%*%A
\end{Sinput}
\begin{Soutput}
     [,1] [,2] [,3]
[1,]    0    0    0
[2,]    0    0    0
[3,]    0    0    0
\end{Soutput}
\begin{Sinput}
> A[,3]<-A[,2]+A[,3]
> A
\end{Sinput}
\begin{Soutput}
     [,1] [,2] [,3]
[1,]    1    1    4
[2,]    5    2    8
[3,]   -2   -1   -4
\end{Soutput}
\end{Schunk}

3.Establecer la \emph{semilla} 13 para cualquier muestra que se necesite.
  \begin{itemize}
  \item Crear una muestra con reemplazo de tamaño 144 tomados del 0 a 358 y guardar en la variable \emph{libros.vendidos}
  \item Con la variable \emph{libros.vendidos} crear una matriz de dimensión 12x12 de tal forma que los elementos se acomoden por filas y guardar la matriz como \emph{libreria}.
  \item Asignar los siguientes nombres a las columnas de \emph{libreria} $2001,2002, \hdots, 2012$
  \item Asignar nombres a las filas: $suc1,\hdots, suc11, suc12$
  
  \end{itemize}

\begin{Schunk}
\begin{Sinput}
> library(xtable)
> set.seed(13)
> libros.vendidos<-sample(0:358,144,TRUE)
> libreria<-matrix(libros.vendidos,12,12,byrow=TRUE)
> colnames(libreria)<-as.character(2001:2012)
> rownames(libreria)<-paste0("suc",1:12)
> libreria<-data.frame(libreria)
> xtable(head(libreria),"6 primeras lineas de libreria")
\end{Sinput}
% latex table generated in R 3.4.0 by xtable 1.8-2 package
% Sat Jun 24 00:10:04 2017
\begin{table}[ht]
\centering
\begin{tabular}{rrrrrrrrrrrrr}
  \hline
 & X2001 & X2002 & X2003 & X2004 & X2005 & X2006 & X2007 & X2008 & X2009 & X2010 & X2011 & X2012 \\ 
  \hline
suc1 & 255 &  88 & 139 &  32 & 345 &   3 & 206 & 274 & 313 &  14 & 237 & 315 \\ 
  suc2 & 319 & 203 & 213 & 130 & 128 & 212 & 310 & 244 &  49 & 196 & 243 & 189 \\ 
  suc3 &  31 & 222 &  11 & 165 & 118 & 227 & 151 & 145 & 329 & 290 & 193 &  38 \\ 
  suc4 & 163 &  18 & 272 &  30 & 216 &  82 & 347 & 217 & 229 & 329 &  53 & 118 \\ 
  suc5 & 262 & 244 & 200 & 142 &  25 & 358 &   7 & 162 &  52 & 122 &  83 &  34 \\ 
  suc6 & 178 & 205 & 287 & 225 & 125 & 162 & 107 &  30 & 140 & 154 &  97 & 105 \\ 
   \hline
\end{tabular}
\caption{6 primeras lineas de libreria} 
\end{table}\end{Schunk}

Suponga que la matriz \emph{libreria} contiene la información acerca de la cantidad de libros de cálculo vendidos para cada año ($2001,2002,\hdots, 2012$) y cada sucursal ($suc1,suc22,\hdots, suc12$). Conteste las siguientes preguntas.

a) Durante los 12 años ¿qué sucursal ha vendido más libros?

\begin{Schunk}
\begin{Sinput}
> which(rowSums(libreria) == max(rowSums(libreria)))
\end{Sinput}
\begin{Soutput}
suc2 
   2 
\end{Soutput}
\begin{Sinput}
> which.max(rowSums(libreria)) #Equivalente a la linea anterior
\end{Sinput}
\begin{Soutput}
suc2 
   2 
\end{Soutput}
\end{Schunk}

b) Durante los 12 años ¿qué sucursal ha vendido menos libros?

\begin{Schunk}
\begin{Sinput}
> which.min(rowSums(libreria))
\end{Sinput}
\begin{Soutput}
suc5 
   5 
\end{Soutput}
\end{Schunk}

c) ¿En qué año se vendieron más libros de cálculo?
\begin{Schunk}
\begin{Sinput}
> which.max(colSums(libreria))
\end{Sinput}
\begin{Soutput}
X2003 
    3 
\end{Soutput}
\end{Schunk}

d) Entre las sucursales 1,5,8 y 12 ¿en total cuántos libros vendieron en 2008?

\begin{Schunk}
\begin{Sinput}
> total08<-sum(libreria[c(1,5,8,12),"2008"])
> print(total08)
\end{Sinput}
\begin{Soutput}
[1] 0
\end{Soutput}
\end{Schunk}

e) ¿En qué año la sucursal 3 vendió más libros de cálculo?
\begin{Schunk}
\begin{Sinput}
> which.max(libreria["suc3",]) 
\end{Sinput}
\begin{Soutput}
X2009 
    9 
\end{Soutput}
\end{Schunk}

\end{document}
